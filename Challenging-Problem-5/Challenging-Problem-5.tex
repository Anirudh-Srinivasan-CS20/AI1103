\documentclass[journal,12pt,twocolumn]{IEEEtran}

\usepackage{setspace}
\usepackage{gensymb}
\singlespacing
\usepackage[cmex10]{amsmath}

\usepackage{amsthm}
\usepackage{amssymb}

\usepackage{mathrsfs}
\usepackage{txfonts}
\usepackage{stfloats}
\usepackage{bm}
\usepackage{cite}
\usepackage{cases}
\usepackage{subfig}

\usepackage{longtable}
\usepackage{multirow}

\usepackage{enumitem}
\usepackage{mathtools}
\usepackage{steinmetz}
\usepackage{tikz}
\usepackage{circuitikz}
\usepackage{verbatim}
\usepackage{tfrupee}
\usepackage[breaklinks=true]{hyperref}
\usepackage{graphicx}
\usepackage{tkz-euclide}

\usetikzlibrary{calc,math}
\usepackage{listings}
    \usepackage{color}                                            %%
    \usepackage{array}                                            %%
    \usepackage{longtable}                                        %%
    \usepackage{calc}                                             %%
    \usepackage{multirow}                                         %%
    \usepackage{hhline}                                           %%
    \usepackage{ifthen}                                           %%
    \usepackage{lscape}     
\usepackage{multicol}
\usepackage{chngcntr}
\usepackage{float}
\restylefloat{table}

\usepackage[utf8]{inputenc}
\usepackage[english]{babel}


\DeclareMathOperator*{\Res}{Res}

\renewcommand\thesection{\arabic{section}}
\renewcommand\thesubsection{\thesection.\arabic{subsection}}
\renewcommand\thesubsubsection{\thesubsection.\arabic{subsubsection}}

\renewcommand\thesectiondis{\arabic{section}}
\renewcommand\thesubsectiondis{\thesectiondis.\arabic{subsection}}
\renewcommand\thesubsubsectiondis{\thesubsectiondis.\arabic{subsubsection}}


\hyphenation{op-tical net-works semi-conduc-tor}
\def\inputGnumericTable{}                                 %%

\lstset{
%language=C,
frame=single, 
breaklines=true,
columns=fullflexible
}
\begin{document}
\newtheorem{theorem}{Theorem}[section]
\newtheorem{problem}{Problem}
\newtheorem{proposition}{Proposition}[section]
\newtheorem{lemma}{Lemma}[section]
\newtheorem{corollary}[theorem]{Corollary}
\newtheorem{example}{Example}[section]
\newtheorem{definition}[problem]{Definition}

\newcommand{\BEQA}{\begin{eqnarray}}
\newcommand{\EEQA}{\end{eqnarray}}
\newcommand{\define}{\stackrel{\triangle}{=}}
\bibliographystyle{IEEEtran}
\raggedbottom
\setlength{\parindent}{0pt}
\providecommand{\mbf}{\mathbf}
\providecommand{\pr}[1]{\ensuremath{\Pr\left(#1\right)}}
\providecommand{\qfunc}[1]{\ensuremath{Q\left(#1\right)}}
\providecommand{\sbrak}[1]{\ensuremath{{}\left[#1\right]}}
\providecommand{\lsbrak}[1]{\ensuremath{{}\left[#1\right.}}
\providecommand{\rsbrak}[1]{\ensuremath{{}\left.#1\right]}}
\providecommand{\brak}[1]{\ensuremath{\left(#1\right)}}
\providecommand{\lbrak}[1]{\ensuremath{\left(#1\right.}}
\providecommand{\rbrak}[1]{\ensuremath{\left.#1\right)}}
\providecommand{\cbrak}[1]{\ensuremath{\left\{#1\right\}}}
\providecommand{\lcbrak}[1]{\ensuremath{\left\{#1\right.}}
\providecommand{\rcbrak}[1]{\ensuremath{\left.#1\right\}}}
\theoremstyle{remark}
\newtheorem{rem}{Remark}
\newcommand{\sgn}{\mathop{\mathrm{sgn}}}
\newcommand*{\permcomb}[4][0mu]{{{}^{#3}\mkern#1#2_{#4}}}
\newcommand*{\perm}[1][-3mu]{\permcomb[#1]{P}}
\newcommand*{\comb}[1][-1mu]{\permcomb[#1]{C}}
\providecommand{\abs}[1]{\vert#1\vert}
\providecommand{\res}[1]{\Res\displaylimits_{#1}} 
\providecommand{\norm}[1]{\lVert#1\rVert}
%\providecommand{\norm}[1]{\lVert#1\rVert}
\providecommand{\mtx}[1]{\mathbf{#1}}
\providecommand{\mean}[1]{E[ #1 ]}
\providecommand{\fourier}{\overset{\mathcal{F}}{ \rightleftharpoons}}
%\providecommand{\hilbert}{\overset{\mathcal{H}}{ \rightleftharpoons}}
\providecommand{\system}{\overset{\mathcal{H}}{ \longleftrightarrow}}
	%\newcommand{\solution}[2]{\textbf{Solution:}{#1}}
\newcommand{\solution}{\noindent \textbf{Solution: }}
\newcommand{\cosec}{\,\text{cosec}\,}
\providecommand{\dec}[2]{\ensuremath{\overset{#1}{\underset{#2}{\gtrless}}}}
\newcommand{\myvec}[1]{\ensuremath{\begin{pmatrix}#1\end{pmatrix}}}
\newcommand{\mydet}[1]{\ensuremath{\begin{vmatrix}#1\end{vmatrix}}}
\numberwithin{equation}{subsection}
\makeatletter
\@addtoreset{figure}{problem}
\makeatother
\let\StandardTheFigure\thefigure
\let\vec\mathbf
\renewcommand{\thefigure}{\theproblem}
\def\putbox#1#2#3{\makebox[0in][l]{\makebox[#1][l]{}\raisebox{\baselineskip}[0in][0in]{\raisebox{#2}[0in][0in]{#3}}}}
     \def\rightbox#1{\makebox[0in][r]{#1}}
     \def\centbox#1{\makebox[0in]{#1}}
     \def\topbox#1{\raisebox{-\baselineskip}[0in][0in]{#1}}
     \def\midbox#1{\raisebox{-0.5\baselineskip}[0in][0in]{#1}}
\vspace{3cm}
\title{AI1103 - Challenging Problem 5}
\author{Anirudh Srinivasan\\CS20BTECH11059}
\maketitle
\newpage
\bigskip
\renewcommand{\thefigure}{\theenumi}
\renewcommand{\thetable}{\theenumi}
Download latex-tikz codes from 
%
\begin{lstlisting}
https://github.com/Anirudh-Srinivasan-CS20/AI1103/blob/main/Challenging-Problem-5//Challenging-Problem-5.tex
\end{lstlisting}
\section*{Question}
Suppose $X_1$,$X_2$,$X_3$ and $X_4$ are independent and identically distributed random variables, having density function f. Then,
\begin{enumerate}[label=\Alph*)]
\item \pr{X_4 > Max(X_1,X_2) > X_3} = $\frac{1}{6}$
\item \pr{X_4 > Max(X_1,X_2) > X_3} = $\frac{1}{8}$
\item \pr{X_4 > X_3 > Max(X_1,X_2)} = $\frac{1}{12}$
\item \pr{X_4 > X_3 > Max(X_1,X_2)} = $\frac{1}{6}$
\end{enumerate}

\section*{Solution}
Given, $X_1,X_2,X_3,X_4$ are i.i.d random variables.
\begin{lemma}
Every i.i.d sequence of random variables is exchangeable. 
Any value of a finite sequence is as likely as any permutation of those values. The joint probability distribution is invariant under the symmetric group.
\end{lemma}

\begin{proof}
\begin{align}
    f_{X_1,X_2,X_3,\dots,X_n}(x) = f_{X_1}(x) \times f_{X_2}(x) \times \dots f_{X_n}(x) 
\end{align}
As $X_i$s are i.i.d random variables, their joint probability density function is the product of their marginal probability density functions and as multiplication is commutative, it is exchangeable.
    
\end{proof}

\begin{definition}[Symmetric Group]
It is the group of permutation on a set with $n$ elements and has $n!$ elements. Order of a symmetric group represents the number of elements in it.
\end{definition}

\begin{lemma}
If $n$ elements of a set are random, then probability of each element '$E_i$' of the symmetric group 'S' is $\frac{1}{n!}$.
\end{lemma}
\begin{proof}
As the $n$ values are completely random, there will be no bias for a particular arrangement and hence all the elements of the symmetric group are equally likely.
\begin{align}
    O(S) = n!
\end{align}
where: O(S) denotes the order of the symmetric group.
\begin{align}
\implies \pr{E_i} = \frac{1}{n!} \forall E_i \in S \label{Pr(E)}
\end{align}

\end{proof}
Hence, any permutation of $X_1,X_2,X_3,X_4$ is equally likely.
As there are 4 random values that the random variables represent, they can be arranged in 4! ways. By \eqref{Pr(E)}, we have:
\begin{align}
   \pr{X_1>X_2>X_3>X_4} &= \pr{X_1>X_2>X_4>X_3} \\&= \dots\\ &= \frac{1}{24}
\end{align}

\begin{enumerate}
    \item \textbf{Options A and B:}
\begin{multline}
    \pr{X_4 > Max(X_1,X_2) > X_3} = \\\pr{X_4 > Max(X_1,X_2) > X_3 | X_1>X2 } + \\ \pr{X_4 > Max(X_1,X_2) > X_3 | X_2>X_1 }
\end{multline}
Clearly, by definition:
\begin{align}
  Max(X_1,X_2) = 
  \begin{cases}
      X_1, & \text{if } X_1>X_2\\
    X_2, & \text{if } X_2>X_1
  \end{cases}
\end{align}
\begin{multline}
    \implies \pr{X_4 > Max(X_1,X_2) > X_3} = \\\pr{X_4 > X_1 > X_3| X_1 > X_2 } + \\ \pr{X_4 > X_2 > X_3| X_2 > X_1}
\end{multline}
\begin{multline}
    \pr{X_4 > X_1 > X_3 | X_1 > X_2 } = \\ \frac{\pr{(X_4 > X_1 > X_3) \cap (X_1 > X_2) }}{\pr{X_1>X_2}} \label{defn_AB}
\end{multline}
$(X_4>X_1) \cap (X_1>X_2) \implies X_4>X_2$ . Hence, imposing the additional condition, we get:
\begin{multline}
    \pr{(X_4 > X_1 > X_3) \cap (X_1 > X_2)} = \\\pr{X_4>X_2} \times \pr{X_4 > X_1 > X_3} \times \pr{X_1 > X_2}
\end{multline}
\begin{multline}
 \pr{(X_4 > X_1 > X_3) \cap (X_1 > X_2)}\\
    = \frac{1}{2!}\times \frac{1}{3!}\times \frac{1}{2!}
    = \frac{1}{2}\times \frac{1}{6}\times \frac{1}{2}= \frac{1}{24} \label{NR_AB}
\end{multline}
\begin{align}
    \pr{X_1>X_2} = \frac{1}{2!} = \frac{1}{2} \label{DR_AB}
\end{align}
Substituting \eqref{NR_AB} and \eqref{DR_AB} in \eqref{defn_AB}:
\begin{align}
    \pr{X_4 > X_1 > X_3 | X_1 > X_2 } = \dfrac{\frac{1}{24}}{\frac{1}{2}} = \frac{1}{12}
\end{align}
\textbf{Aliter:}\\
The event $(X_4 > X_1 > X_3 | X_1 > X_2)$ can be decomposed into its constituent sub-events and hence we have:
\begin{multline}
    \pr{X_4 > X_1 > X_3 | X_1 > X_2 } = \\\pr{X_4 > X_1 > X_2> X_3} \\+\pr{X_4 > X_1 > X_3 > X_2}
= \frac{1}{12}
\end{multline}
As $Max(X_1,X_2)$ being $X_1$ or $X_2$ is equally likely,
\begin{align}
\pr{X_4 > X_2 > X_3| X_2 > X_1} = \frac{1}{12}
\end{align}
\begin{align}
\pr{X_4 > Max(X_1,X_2) > X_3} = 2 \times \frac{1}{12} = \frac{1}{6}
\end{align}

\item \textbf{Options C and D:} 

\begin{multline}
    \pr{X_4 > X_3 > Max(X_1,X_2)} = \\\pr{X_4 > X_3 > Max(X_1,X_2)| X_1>X_2} + \\\pr{X_4 > X_3 > Max(X_1,X_2)| X_2>X_1}
\end{multline}
\begin{multline}
    \pr{X_4 > X_3 > Max(X_1,X_2)} = \\\pr{X_4 > X_3 > X_1| X_1>X_2} + \\\pr{X_4 > X_3 > X_2| X_2>X_1}
\end{multline}
\begin{multline}
    \pr{X_4 > X_3 > X_1 | X_1 > X_2 } = \\ \frac{\pr{(X_4 > X_3 > X_1) \cap (X_1 > X_2) }}{\pr{X_1>X_2}} \label{defn_CD}
\end{multline}
 $(X_4>X_1) \cap (X_1>X_2) \implies X_4>X_2$ and $(X_3>X_1) \cap (X_1>X_2) \implies X_3>X_2$ . Hence, Hence, imposing the additional conditions, we get:
\begin{multline}
    \pr{(X_4 > X_3 > X_1) \cap (X_1 > X_2)} = \pr{X_3>X_2} \times \\\pr{X_4>X_2} \times \pr{X_4 > X_1 > X_3} \times \pr{X_1 > X_2}
\end{multline}
\begin{multline}
 \pr{(X_4 > X_3 > X_1) \cap (X_1 > X_2)}\\
    = \frac{1}{2!}\times \frac{1}{2!}\times \frac{1}{3!}\times \frac{1}{2!}
    = \frac{1}{2}\times \frac{1}{2}\times \frac{1}{6}\times \frac{1}{2}= \frac{1}{48} \label{NR_CD}
\end{multline}
\begin{align}
    \pr{X_1>X_2} = \frac{1}{2!} = \frac{1}{2} \label{DR_CD}
\end{align}
Substituting \eqref{NR_CD} and \eqref{DR_CD} in \eqref{defn_CD}:
\begin{align}
    \pr{X_4 > X_3 > X_1 | X_1 > X_2 } = \dfrac{\frac{1}{48}}{\frac{1}{2}} = \frac{1}{24}
\end{align}
\textbf{Aliter:}\\
The event $(X_4 > X_3 > X_1 | X_1 > X_2)$ can be decomposed into its constituent sub-events and hence we have:
\begin{multline}
    \pr{X_4 > X_3 > X_1|X_1>X_2} \\= \pr{X_4 > X_3 > X_1 > X_2} = \frac{1}{24}
\end{multline}
As $Max(X_1,X_2)$ being $X_1$ or $X_2$ is equally likely,
\begin{align}
\pr{X_4 > X_3 > X_2| X_2 > X_1} = \frac{1}{24}
\end{align}
\begin{align}
    \pr{X_4 > X_3 > Max(X_1,X_2)} = 2 \times \frac{1}{24} = \frac{1}{12}
\end{align}
\end{enumerate}

\bigskip
\rightline{Answer: Option (A) and (C)}

\end{document}