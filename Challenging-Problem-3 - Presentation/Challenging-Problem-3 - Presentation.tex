\documentclass{beamer}
\usepackage{listings}
\lstset{
%language=C,
frame=single, 
breaklines=true,
columns=fullflexible
}
\usepackage{subcaption}
\usepackage{url}
\usepackage{tikz}
\usepackage{tkz-euclide} % loads  TikZ and tkz-base
%\usetkzobj{all}
\usetikzlibrary{calc,math}
\usepackage{float}
\newcommand\norm[1]{\left\lVert#1\right\rVert}
\providecommand{\brak}[1]{\ensuremath{\left(#1\right)}}
\providecommand{\abs}[1]{\vert#1\vert}
\providecommand{\fourier}{\overset{\mathcal{F}}{ \rightleftharpoons}}
\providecommand{\pr}[1]{\ensuremath{\Pr\left(#1\right)}}
\providecommand{\sbrak}[1]{\ensuremath{{}\left[#1\right]}}
\renewcommand{\vec}[1]{\mathbf{#1}}
\usepackage[export]{adjustbox}
\usepackage[utf8]{inputenc}
\usepackage{amsmath}
\usepackage{mhchem}
\usetheme{Boadilla}

\title{Challenging Problem - 3}
\author{Anirudh Srinivasan}
\institute{IIT Hyderabad}
\date{July 2, 2021}
\begin{document}


\begin{frame}
\titlepage
\end{frame}
\begin{frame}{Problem Definition}
\begin{block}{Question}
Let $X_1, X_2, X_3, X_4, X_5$ be i.i.d random variables having a continuous distribution function. Given: $X_1 = max(X_1,X_2,X_3,X_4,X_5)$. Then, the value of p = $\pr{X_1 > X_2 > X_3 > X_4 > X_5}$ is:

\begin{enumerate}
     \item $\frac{1}{4}$
     \item $\frac{1}{5}$
     \item $\frac{1}{4!}$
     \item $\frac{1}{5!}$
\end{enumerate}
\end{block}
\end{frame}

\begin{frame}{Key Concepts and Definitions}
\begin{block}{Exchangeable random variables}
An exchangeable sequence of random variables (also known as interchangeable) is a sequence $X_1, X_2, X_3, \dots $ (which may be finitely or infinitely long) whose joint probability distribution does not change when the positions in the sequence in which finitely many of them appear are altered. Thus, for example the sequences
\begin{align}
    X_{1},X_{2},X_{3},X_{4},X_{5},X_{6}{\text{ and }}\quad X_{3},X_{6},X_{1},X_{5},X_{2},X_{4}
\end{align}

both have the same joint probability distribution.
\end{block}    


\end{frame}

\begin{frame}{Key Concepts and Definitions Contd.}
\begin{block}{Lemma}
Every i.i.d sequence of random variables is exchangeable. 
Any value of a finite sequence is as likely as any permutation of those values. The joint probability distribution is invariant under the symmetric group.
\end{block}    
\begin{block}{Proof}
\begin{align}
    f_{X_1,X_2,X_3,\dots,X_n}(x) = f_{X_1}(x) \times f_{X_2}(x) \times \dots f_{X_n}(x) 
\end{align}
As $X_i$s are i.i.d random variables, their joint probability density function is the product of their marginal probability density functions and as multiplication is commutative, it is exchangeable.
\end{block}

\end{frame}

\begin{frame}{Key Concepts and Definitions Contd.}
\begin{block}{Symmetric group}
It is the group of permutation on a set with $n$ elements and has $n!$ elements. Order of a symmetric group represents the number of elements in it.
\end{block} 
\begin{block}{Lemma}
If $n$ elements of a set are random, then probability of each element '$E_i$' of the symmetric group 'S' is $\frac{1}{n!}$.
\end{block}

\end{frame}

\begin{frame}{Key Concepts and Definitions Contd.}
\begin{block}{Proof}
As the $n$ values are completely random, there will be no bias for a particular arrangement and hence all the elements of the symmetric group are equally likely.
\begin{align}
    O(S) = n!
\end{align}
where: O(S) denotes the order of the symmetric group.
\begin{align}
\implies \pr{E_i} = \frac{1}{n!} \forall E_i \in S \label{Pr(E)}
\end{align}
\end{block}
\end{frame}
\begin{frame}{Solution}
\begin{block}{}
Given, $X_1,X_2,X_3,X_4,X_5$ are i.i.d random variables.
Clearly, in the conditional world where: $X_1 = max(X_1,X_2,X_3,X_4,X_5)$, we have:
\begin{align}
 p = \pr{X_2 > X_3 > X_4 > X_5}
\end{align}
\end{block}
\begin{block}{}
As any permutation of $X_2,X_3,X_4,X_5$ is equally likely.
As there are 4 random values that the random variables represent, they can be arranged in 4! ways. By \eqref{Pr(E)}, we get:
\begin{align}
   \pr{X_2>X_3>X_4>X_5} &= \frac{1}{4!}\\&= \frac{1}{24}
\end{align}
\end{block}
\begin{block}{}
 Hence, the answer is Option (C)
\end{block}
\end{frame}

\end{document}