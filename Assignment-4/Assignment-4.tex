\documentclass[journal,12pt,twocolumn]{IEEEtran}

\usepackage{setspace}
\usepackage{gensymb}
\singlespacing
\usepackage[cmex10]{amsmath}

\usepackage{amsthm}
\usepackage{amssymb}

\usepackage{mathrsfs}
\usepackage{txfonts}
\usepackage{stfloats}
\usepackage{bm}
\usepackage{cite}
\usepackage{cases}
\usepackage{subfig}

\usepackage{longtable}
\usepackage{multirow}

\usepackage{enumitem}
\usepackage{mathtools}
\usepackage{steinmetz}
\usepackage{tikz}
\usepackage{circuitikz}
\usepackage{verbatim}
\usepackage{tfrupee}
\usepackage[breaklinks=true]{hyperref}
\usepackage{graphicx}
\usepackage{tkz-euclide}

\usetikzlibrary{calc,math}
\usepackage{listings}
    \usepackage{color}                                            %%
    \usepackage{array}                                            %%
    \usepackage{longtable}                                        %%
    \usepackage{calc}                                             %%
    \usepackage{multirow}                                         %%
    \usepackage{hhline}                                           %%
    \usepackage{ifthen}                                           %%
    \usepackage{lscape}     
\usepackage{multicol}
\usepackage{chngcntr}
\usepackage{float}
\restylefloat{table}

\DeclareMathOperator*{\Res}{Res}

\renewcommand\thesection{\arabic{section}}
\renewcommand\thesubsection{\thesection.\arabic{subsection}}
\renewcommand\thesubsubsection{\thesubsection.\arabic{subsubsection}}

\renewcommand\thesectiondis{\arabic{section}}
\renewcommand\thesubsectiondis{\thesectiondis.\arabic{subsection}}
\renewcommand\thesubsubsectiondis{\thesubsectiondis.\arabic{subsubsection}}


\hyphenation{op-tical net-works semi-conduc-tor}
\def\inputGnumericTable{}                                 %%

\lstset{
%language=C,
frame=single, 
breaklines=true,
columns=fullflexible
}
\begin{document}

\newcommand{\BEQA}{\begin{eqnarray}}
\newcommand{\EEQA}{\end{eqnarray}}
\newcommand{\define}{\stackrel{\triangle}{=}}
\bibliographystyle{IEEEtran}
\raggedbottom
\setlength{\parindent}{0pt}
\providecommand{\mbf}{\mathbf}
\providecommand{\pr}[1]{\ensuremath{\Pr\left(#1\right)}}
\providecommand{\qfunc}[1]{\ensuremath{Q\left(#1\right)}}
\providecommand{\sbrak}[1]{\ensuremath{{}\left[#1\right]}}
\providecommand{\lsbrak}[1]{\ensuremath{{}\left[#1\right.}}
\providecommand{\rsbrak}[1]{\ensuremath{{}\left.#1\right]}}
\providecommand{\brak}[1]{\ensuremath{\left(#1\right)}}
\providecommand{\lbrak}[1]{\ensuremath{\left(#1\right.}}
\providecommand{\rbrak}[1]{\ensuremath{\left.#1\right)}}
\providecommand{\cbrak}[1]{\ensuremath{\left\{#1\right\}}}
\providecommand{\lcbrak}[1]{\ensuremath{\left\{#1\right.}}
\providecommand{\rcbrak}[1]{\ensuremath{\left.#1\right\}}}
\theoremstyle{remark}
\newtheorem{rem}{Remark}
\newcommand{\sgn}{\mathop{\mathrm{sgn}}}
\newcommand*{\permcomb}[4][0mu]{{{}^{#3}\mkern#1#2_{#4}}}
\newcommand*{\perm}[1][-3mu]{\permcomb[#1]{P}}
\newcommand*{\comb}[1][-1mu]{\permcomb[#1]{C}}
\providecommand{\abs}[1]{\vert#1\vert}
\providecommand{\res}[1]{\Res\displaylimits_{#1}} 
\providecommand{\norm}[1]{\lVert#1\rVert}
%\providecommand{\norm}[1]{\lVert#1\rVert}
\providecommand{\mtx}[1]{\mathbf{#1}}
\providecommand{\mean}[1]{E[ #1 ]}
\providecommand{\fourier}{\overset{\mathcal{F}}{ \rightleftharpoons}}
%\providecommand{\hilbert}{\overset{\mathcal{H}}{ \rightleftharpoons}}
\providecommand{\system}{\overset{\mathcal{H}}{ \longleftrightarrow}}
	%\newcommand{\solution}[2]{\textbf{Solution:}{#1}}
\newcommand{\solution}{\noindent \textbf{Solution: }}
\newcommand{\cosec}{\,\text{cosec}\,}
\providecommand{\dec}[2]{\ensuremath{\overset{#1}{\underset{#2}{\gtrless}}}}
\newcommand{\myvec}[1]{\ensuremath{\begin{pmatrix}#1\end{pmatrix}}}
\newcommand{\mydet}[1]{\ensuremath{\begin{vmatrix}#1\end{vmatrix}}}
\numberwithin{equation}{subsection}
\makeatletter
\@addtoreset{figure}{problem}
\makeatother
\let\StandardTheFigure\thefigure
\let\vec\mathbf
\renewcommand{\thefigure}{\theproblem}
\def\putbox#1#2#3{\makebox[0in][l]{\makebox[#1][l]{}\raisebox{\baselineskip}[0in][0in]{\raisebox{#2}[0in][0in]{#3}}}}
     \def\rightbox#1{\makebox[0in][r]{#1}}
     \def\centbox#1{\makebox[0in]{#1}}
     \def\topbox#1{\raisebox{-\baselineskip}[0in][0in]{#1}}
     \def\midbox#1{\raisebox{-0.5\baselineskip}[0in][0in]{#1}}
\vspace{3cm}
\title{AI1103 - Assignment 4}
\author{Anirudh Srinivasan\\CS20BTECH11059}
\maketitle
\newpage
\bigskip
\renewcommand{\thefigure}{\theenumi}
\renewcommand{\thetable}{\theenumi}
Download latex-tikz codes from 
%
\begin{lstlisting}
https://github.com/Anirudh-Srinivasan-CS20/AI1103/blob/main/Assignment-4/Assignment-4.tex
\end{lstlisting}
\section*{Question}
Let $X_1,X_2,\dots$ be independent random variables with $X_n$ being uniformly distributed between -n and 3n, n=1,2,\dots. 
Let $S_N = \frac{1}{\sqrt{N}} \sum_{n=1}^N \frac{X_n}{n}$ for N=1,2,\dots and let $F_N$ be the distribution function of $S_N$. Also, let $\phi$ denote the distribution function of a standard normal variable. Which of the following is/are true? 
\begin{enumerate}[label=\Alph*)]
    \item $\lim_{N\to\infty} F_N(0)\leq \phi(0)$
    \item $\lim_{N\to\infty} F_N(0)\geq \phi(0)$
    \item $\lim_{N\to\infty} F_N(1)\leq \phi(1)$
    \item $\lim_{N\to\infty} F_N(1)\geq \phi(1)$
\end{enumerate}

\section*{Solution}
Given, $X_1,X_2,\dots$ are independent random variables with $X_i\sim \mathcal U(-i,3i)$. Let us define:
\begin{align}
    Y_i = \frac{X_i}{i}\text{ }\forall\text{ i} \\
    \implies Y_i \sim \mathcal U(-1,3)
\end{align}
Now, $Y_1,Y_2,\dots$ are i.i.d (independent and identically distributed) random variables with:
\begin{align}
    \mu = \frac{-1 + 3}{2} = 1\\
    \sigma^2 = \frac{[3 - (-1)]^2}{12} = \frac{16}{12} = \frac{4}{3}
\end{align}
And, we have:
\begin{align}
  S_N &= \frac{1}{\sqrt{N}} \sum_{n=1}^N \frac{X_n}{n} \\
      &= \frac{1}{\sqrt{N}} \sum_{n=1}^N Y_n
\end{align}
By Central Limit Theorem, we can conclude:
\begin{align}
\lim_{N\to\infty} S_N \sim \mathcal N(\mu,\sigma^2)\\
\implies \lim_{N\to\infty} F_N(x) &= \phi\left(\frac{x-\mu}{\sigma}\right)\\
&= \phi\left(\frac{\sqrt{3}(x-1)}{2}\right)
\end{align}
Substituting x=1, we get:
\begin{align}
    \lim_{N\to\infty} F_N(1) &=  \phi\left(\frac{\sqrt{3}(1-1)}{2}\right)\\
    &= \phi(0)
\end{align}
The distribution function $F_N$ is non-decreasing. So,
\begin{align}
    \lim_{N\to\infty} F_N(0) \leq \lim_{N\to\infty} F_N(1) = \phi(0)\\
    \lim_{N\to\infty} F_N(0) \leq \phi(0)
\end{align}

The distribution function $\phi$ is non-decreasing. So,
\begin{align}
    \lim_{N\to\infty} F_N(1) = \phi(0) \leq \phi(1)\\
    \lim_{N\to\infty} F_N(1) \leq \phi(1)
\end{align}
\bigskip
\rightline{Answer: Option (A) and (C)}

\end{document}