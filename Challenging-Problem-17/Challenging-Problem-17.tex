\documentclass[journal,12pt,twocolumn]{IEEEtran}

\usepackage{setspace}
\usepackage{gensymb}
\singlespacing
\usepackage[cmex10]{amsmath}

\usepackage{amsthm}
\usepackage{amssymb}

\usepackage{mathrsfs}
\usepackage{txfonts}
\usepackage{stfloats}
\usepackage{bm}
\usepackage{cite}
\usepackage{cases}
\usepackage{subfig}

\usepackage{longtable}
\usepackage{multirow}

\usepackage{enumitem}
\usepackage{mathtools}
\usepackage{steinmetz}
\usepackage{tikz}
\usepackage{circuitikz}
\usepackage{verbatim}
\usepackage{tfrupee}
\usepackage[breaklinks=true]{hyperref}
\usepackage{graphicx}
\usepackage{tkz-euclide}

\usetikzlibrary{calc,math}
\usepackage{listings}
    \usepackage{color}                                            %%
    \usepackage{array}                                            %%
    \usepackage{longtable}                                        %%
    \usepackage{calc}                                             %%
    \usepackage{multirow}                                         %%
    \usepackage{hhline}                                           %%
    \usepackage{ifthen}                                           %%
    \usepackage{lscape}     
\usepackage{multicol}
\usepackage{chngcntr}
\usepackage{float}
\restylefloat{table}

\usepackage[utf8]{inputenc}
\usepackage[english]{babel}

\newtheorem{theorem}{Theorem}[section]
\newtheorem{corollary}{Corollary}[theorem]
\newtheorem{lemma}[theorem]{Lemma}
\theoremstyle{definition}
\newtheorem{definition}{Definition}[section]

\DeclareMathOperator*{\Res}{Res}

\renewcommand\thesection{\arabic{section}}
\renewcommand\thesubsection{\thesection.\arabic{subsection}}
\renewcommand\thesubsubsection{\thesubsection.\arabic{subsubsection}}

\renewcommand\thesectiondis{\arabic{section}}
\renewcommand\thesubsectiondis{\thesectiondis.\arabic{subsection}}
\renewcommand\thesubsubsectiondis{\thesubsectiondis.\arabic{subsubsection}}


\hyphenation{op-tical net-works semi-conduc-tor}
\def\inputGnumericTable{}                                 %%

\lstset{
%language=C,
frame=single, 
breaklines=true,
columns=fullflexible
}
\begin{document}

\newcommand{\BEQA}{\begin{eqnarray}}
\newcommand{\EEQA}{\end{eqnarray}}
\newcommand{\define}{\stackrel{\triangle}{=}}
\bibliographystyle{IEEEtran}
\raggedbottom
\setlength{\parindent}{0pt}
\providecommand{\mbf}{\mathbf}
\providecommand{\pr}[1]{\ensuremath{\Pr\left(#1\right)}}
\providecommand{\qfunc}[1]{\ensuremath{Q\left(#1\right)}}
\providecommand{\sbrak}[1]{\ensuremath{{}\left[#1\right]}}
\providecommand{\lsbrak}[1]{\ensuremath{{}\left[#1\right.}}
\providecommand{\rsbrak}[1]{\ensuremath{{}\left.#1\right]}}
\providecommand{\brak}[1]{\ensuremath{\left(#1\right)}}
\providecommand{\lbrak}[1]{\ensuremath{\left(#1\right.}}
\providecommand{\rbrak}[1]{\ensuremath{\left.#1\right)}}
\providecommand{\cbrak}[1]{\ensuremath{\left\{#1\right\}}}
\providecommand{\lcbrak}[1]{\ensuremath{\left\{#1\right.}}
\providecommand{\rcbrak}[1]{\ensuremath{\left.#1\right\}}}
\theoremstyle{remark}
\newtheorem{rem}{Remark}
\newcommand{\sgn}{\mathop{\mathrm{sgn}}}
\newcommand*{\permcomb}[4][0mu]{{{}^{#3}\mkern#1#2_{#4}}}
\newcommand*{\perm}[1][-3mu]{\permcomb[#1]{P}}
\newcommand*{\comb}[1][-1mu]{\permcomb[#1]{C}}
\providecommand{\abs}[1]{\vert#1\vert}
\providecommand{\res}[1]{\Res\displaylimits_{#1}} 
\providecommand{\norm}[1]{\lVert#1\rVert}
%\providecommand{\norm}[1]{\lVert#1\rVert}
\providecommand{\mtx}[1]{\mathbf{#1}}
\providecommand{\mean}[1]{E[ #1 ]}
\providecommand{\fourier}{\overset{\mathcal{F}}{ \rightleftharpoons}}
%\providecommand{\hilbert}{\overset{\mathcal{H}}{ \rightleftharpoons}}
\providecommand{\system}{\overset{\mathcal{H}}{ \longleftrightarrow}}
	%\newcommand{\solution}[2]{\textbf{Solution:}{#1}}
\newcommand{\solution}{\noindent \textbf{Solution: }}
\newcommand{\cosec}{\,\text{cosec}\,}
\providecommand{\dec}[2]{\ensuremath{\overset{#1}{\underset{#2}{\gtrless}}}}
\newcommand{\myvec}[1]{\ensuremath{\begin{pmatrix}#1\end{pmatrix}}}
\newcommand{\mydet}[1]{\ensuremath{\begin{vmatrix}#1\end{vmatrix}}}
\numberwithin{equation}{subsection}
\makeatletter
\@addtoreset{figure}{problem}
\makeatother
\let\StandardTheFigure\thefigure
\let\vec\mathbf
\renewcommand{\thefigure}{\theproblem}
\def\putbox#1#2#3{\makebox[0in][l]{\makebox[#1][l]{}\raisebox{\baselineskip}[0in][0in]{\raisebox{#2}[0in][0in]{#3}}}}
     \def\rightbox#1{\makebox[0in][r]{#1}}
     \def\centbox#1{\makebox[0in]{#1}}
     \def\topbox#1{\raisebox{-\baselineskip}[0in][0in]{#1}}
     \def\midbox#1{\raisebox{-0.5\baselineskip}[0in][0in]{#1}}
\vspace{3cm}
\title{AI1103 - Challenging Problem 17}
\author{Anirudh Srinivasan\\CS20BTECH11059}
\maketitle
\newpage
\bigskip
\renewcommand{\thefigure}{\theenumi}
\renewcommand{\thetable}{\theenumi}
Download latex-tikz codes from 
%
\begin{lstlisting}
https://github.com/Anirudh-Srinivasan-CS20/AI1103/blob/main/Challenging-Problem-17//Challenging-Problem-17.tex
\end{lstlisting}
\section*{Question}
 Let X and Y be independent and identically distributed random variables such that \(\pr{X=0}=\pr{X=1}= \frac{1}{2}\). Let \(Z = X+Y \text{ and } W = \abs{X-Y}\). Then which statement is not correct?
 \begin{enumerate}[label=\Alph*)]
     \item $X \text{ and } W$ are independent.
     \item $Y \text{ and } W$ are independent.
     \item $Z \text{ and } W$ are uncorrelated.
     \item $Z \text{ and } W$ are independent.
 \end{enumerate}

\section*{Solution}
Since, X and Y are i.i.d random variables
\begin{align}
    \pr{X = 0} = \pr{Y = 0} = \frac{1}{2}\\
    \pr{X = 1} = \pr{Y = 1} = \frac{1}{2}
\end{align}
\begin{table}[H]
\centering
\begin{tabular}{|c|c|c|c|c|c|}
\hline
X & Y & Z & W & ZW  &\pr{X,Y} \\ \hline
0 & 0 & 0 & 0 & 0   & $\frac{1}{4}$     \\ \hline
0 & 1 & 1 & 1 & 1   & $\frac{1}{4}$       \\ \hline
1 & 0 & 1 & 1 & 1   & $\frac{1}{4}$        \\ \hline
1 & 1 & 2 & 0 & 0   & $\frac{1}{4}$       \\ \hline
\end{tabular}
\caption{This table shows probability associated with each value that the random variable Z,W and ZW can take when X = x, Y = y, Z = x+y, W = $\abs{x-y}$ and ZW = (x+y)$\abs{x-y}$. }
\label{tab:Table 17.10}
\end{table}
$\implies$ For Z, we have:
\begin{align}
    \pr{Z = 0} =\frac{1}{4}\\
    \pr{Z = 1} =\frac{1}{2}\\
    \pr{Z = 2} =\frac{1}{4}
\end{align}
\begin{align}
E(Z) &= \sum_{i = 0}^{2} Pr(Z=i) \times z_{i}\\
&= \frac{1}{4} \times 0 + \frac{1}{2} \times 1 + \frac{1}{4} \times 2\\
&= 1
\end{align}
$\implies$ For W, we have:
\begin{align}
    \pr{W = 0} =\frac{1}{2}\\
    \pr{W = 1} =\frac{1}{2}
\end{align}
\begin{align}
E(W) &= \sum_{i = 0}^{1} Pr(W=i) \times w_{i}\\
&= \frac{1}{2} \times 0 + \frac{1}{2} \times 1\\
&= \frac{1}{2}
\end{align}
$\implies$ For ZW, we have:
\begin{align}
    \pr{ZW = 0} =\frac{1}{2}\\
    \pr{ZW = 1} =\frac{1}{2}
\end{align}
\begin{align}
E(ZW) &= \sum_{i = 0}^{1} Pr(WZ=i) \times (wz)_{i}\\
&= \frac{1}{2} \times 0 + \frac{1}{2} \times 1\\
&= \frac{1}{2}
\end{align}
\begin{align}
    E(ZW) = \frac{1}{2}\\
    E(Z)\times E(W) = 1 \times \frac{1}{2}\\
    E(ZW) = E(Z)\times E(W)\\
    cov(ZW) = E(ZW) - E(Z)\times E(W) = 0 
\end{align}
    The correlation coefficient between Z and W is given by:
\begin{align}
\rho_{Z,W} &= \frac{cov(ZW)}{\sigma_z\times \sigma_w}\\
&= 0
\end{align}
\begin{center}
   $\implies$ Z and W are uncorrelated.
\end{center}
From the above table:
\begin{align}
    \pr{X = 0 | W = 0} = \frac{1}{2} = \pr{X = 0}\\
    \pr{X = 0 | W = 1} = \frac{1}{2} = \pr{X = 0}\\
    \pr{X = 1 | W = 0} = \frac{1}{2} = \pr{X = 1}\\
    \pr{X = 1 | W = 1} = \frac{1}{2} = \pr{X = 1}
\end{align}
\begin{center}
   $\implies$ X and W are independent. 
\end{center}
As the distribution of X and Y are identical, similarly we get:
\begin{center}
   $\implies$ Y and W are independent. 
\end{center}
\begin{align}
    \pr{Z = 0 | W = 0} = \frac{1}{2} \neq \pr{Z = 0}
\end{align}
\begin{center}
   $\implies$ Z and W are dependent. 
\end{center}
\bigskip
\rightline{Answer: Option (D)}

\end{document}